% Copyright 2004 by Till Tantau <tantau@users.sourceforge.net>.
%
% In principle, this file can be redistributed and/or modified under
% the terms of the GNU Public License, version 2.
%
% However, this file is supposed to be a template to be modified
% for your own needs. For this reason, if you use this file as a
% template and not specifically distribute it as part of a another
% package/program, I grant the extra permission to freely copy and
% modify this file as you see fit and even to delete this copyright
% notice. 

\documentclass[10pt]{beamer}

% There are many different themes available for Beamer. A comprehensive
% list with examples is given here:
% http://deic.uab.es/~iblanes/beamer_gallery/index_by_theme.html
% You can uncomment the themes below if you would like to use a different
% one:
%\usetheme{AnnArbor}
%\usetheme{Antibes}
%\usetheme{Bergen}
%\usetheme{Berkeley}
%\usetheme{Berlin}
\usetheme{Boadilla}
%\usetheme{boxes}
%\usetheme{CambridgeUS}
%\usetheme{Copenhagen}
%\usetheme{Darmstadt}
%\usetheme{default}
%\usetheme{Frankfurt}
%\usetheme{Goettingen}
%\usetheme{Hannover}
%\usetheme{Ilmenau}
%\usetheme{JuanLesPins}
%\usetheme{Luebeck}
%\usetheme{Madrid}
%\usetheme{Malmoe}
%\usetheme{Marburg}
%\usetheme{Montpellier}
%\usetheme{PaloAlto}
%\usetheme{Pittsburgh}
%\usetheme{Rochester}
%\usetheme{Singapore}
%\usetheme{Szeged}
%\usetheme{Warsaw}
%\usetheme{metropolis}
%\usetheme[numbering=none,progressbar=frametitle,block=fill]{metropolis}
%    \setbeamertemplate{frametitle}
%      {\begin{centering}\smallskip
%      \insertframetitle\par
%       \smallskip\end{centering}}
%    \setbeamertemplate{itemize item}{$\bullet$}
%    \setbeamertemplate{navigation symbols}{}
%    \setbeamertemplate{footline}[text line]{%
%        \hfill\strut{%
%            \scriptsize\sf\color{black!60}%
%            \quad\insertframenumber
%        }%
%        \hfill
%    }

    % Define some colors:
%    \definecolor{DarkFern}{HTML}{407428}
%    \definecolor{DarkCharcoal}{HTML}{4D4944}
%    \colorlet{Fern}{DarkFern!85!white}
%    \colorlet{Charcoal}{DarkCharcoal!110!white}
%    \colorlet{LightCharcoal}{Charcoal!50!white}
%    \colorlet{AlertColor}{orange!80!black}
%    \colorlet{DarkRed}{red!70!black}
%    \colorlet{DarkBlue}{blue!70!black}
%    \colorlet{DarkGreen}{green!70!black}
     
%    % Use the colors:
%    \setbeamercolor{title}{fg=Fern}
%    \setbeamercolor{frametitle}{fg=Fern}
%    \setbeamercolor{normal text}{fg=Charcoal}
%    \setbeamercolor{block title}{fg=black,bg=Fern!30!white}
%    \setbeamercolor{block body}{fg=black,bg=Fern!20!white}
%    \setbeamercolor{alerted text}{fg=AlertColor}
%    \setbeamercolor{itemize item}{fg=Charcoal}

\usecolortheme{mines}

\setbeamercovered{dynamic}

\usepackage[utf8]{inputenc}
\usepackage{amsmath}
\usepackage{amsfonts}
\usepackage{amssymb}
\usepackage{graphicx}
\usepackage{hyperref}
\usepackage{multimedia}
\usepackage{stackengine}
\usepackage{media9}
\usepackage{tipa}


\setbeamersize{text margin left=1cm,text margin right=1cm} 

\title[Emmy Noether]{The Great Mathematician Emmy Noether}

% A subtitle is optional and this may be deleted
\subtitle{(1882--1935)}

%\titlegraphic{\includegraphics{Noether}}

\author[J. McKinsey]{\includegraphics[scale=0.8]{Noether} \\ \vspace{1em} Joseph
  McKinsey \vspace{-2em}}
% - Give the names in the same order as the appear in the paper.
% - Use the \inst{?} command only if the authors have different
%   affiliation.

\institute[Colorado School of Mines] % (optional, but mostly needed)
%{
%  \inst{1}%
%  Department of Computer Science\\
%  University of Somewhere
%  \and
%  \inst{2}%
%  Department of Theoretical Philosophy\\
%  University of Elsewhere}
% - Use the \inst command only if there are several affiliations.
% - Keep it simple, no one is interested in your street address.

\date{\today}
% - Either use conference name or its abbreviation.
% - Not really informative to the audience, more for people (including
%   yourself) who are reading the slides online

\subject{Famous Mathematicians}
% This is only inserted into the PDF information catalog. Can be left
% out. 

% If you have a file called "university-logo-filename.xxx", where xxx
% is a graphic format that can be processed by latex or pdflatex,
% resp., then you can add a logo as follows:

%\pgfdeclareimage[height=0.5cm]{university-logo}{Noether}
%\logo{\pgfuseimage{university-logo}}

% Delete this, if you do not want the table of contents to pop up at
% the beginning of each subsection:
%\AtBeginSubsection[]
%{
%  \begin{frame}<beamer>{Outline}
%    \tableofcontents[currentsection,currentsubsection]
%  \end{frame}
%}

% Let's get started
\begin{document}

\begin{frame}
  \titlepage
\end{frame}

\begin{frame}{Overview}
  \tableofcontents
  % You might wish to add the option [pausesections]
\end{frame}

% Section and subsections will appear in the presentation overview
% and table of contents.
\section{Introduction}
\begin{frame}{Introduction}
Emmy Noether was a German mathematician in G\"ottingen University in the early
1900s.

\begin{alertblock}{Pronunciation}
\begin{center} % IPA is trash.
\large \begin{IPA}
'n\o:t5
\end{IPA}
\end{center}
Not as in ``another''. Pronounce it like neuter, but with a long ``u'' sound. \cite{wikipedia}
\end{alertblock}

\begin{block}{Contributions}
\begin{enumerate}
\item Abstract mathematics
\item Noncommunicable algebra
\item Theoretical physics
\end{enumerate}
\end{block}

Described as ``fat, rough, and loud, but so kind, humorous and sociable that all
know new her loved her'' \cite{biography}.
\end{frame}


\section{Biography}
\subsection{Early Life}
\begin{frame}{Early Life}
\begin{columns}
\begin{column}{0.6\textwidth}
\begin{itemize}
\item Born in Erlangen, Germany on March 23rd, 1882, to a Jewish family.
\item Not a child genius of prodigy.
\item Two slightly younger brothers and one much younger brother.
\end{itemize}
\begin{figure}[h]
    \centering
    \includegraphics[width=0.55\textwidth]{family}
    \caption{Noether, Emmy; Noether, Alfred; Noether, Fritz; Noether, Robert --
Annotation: vor 1918 -- Source: Konrad Jacobs, Erlangen \cite{wikipedia}. }
\end{figure}
\end{column}
\begin{column}{0.4\textwidth}

\includegraphics[width=\textwidth]{Location}
\scriptsize Modified from Map by Karte: NordNordWest, Lizenz: Creative Commons by-sa-3.0 de, CC BY-SA 3.0 de, \url{https://commons.wikimedia.org/w/index.php?curid=35392837}
\end{column}
\end{columns}
\end{frame}

\subsection{University Education}
\begin{frame}{University Education}
\begin{itemize}
\item Since woman were unable to attend, Noether had to audit classes at the
university of Erlangen.
\item She graduated from a \textit{Realgymnasium} in Nuremberg \cite{wikipedia}.
\end{itemize}
\begin{block}{Graduate Program}
\begin{itemize}
\item Noether returned to the University of Erlangen after the prohibition was lifted.
\item In 1907, she wrote her dissertation ``\"Uber die Bildung des Formensystems
der tern\"aren biquadratischen Form'' (On Complete Systems of Invariants for Ternary Biquadratic Forms, 1907) \cite{biography}.
\item For her dissertation, over \(300\) different forms of invariance were
listed as her supervisor suggested to her. At the same time, David Hilbert's
work was making this expensive process largely irrelevant.
\end{itemize}
\end{block}
\end{frame}

\subsection{Professorship}
\begin{frame}{Teaching at the University of Erlangen}
\begin{itemize}
\item From 1908 to 1915, she taught at the University of Erlangen without pay. Noether was introduced to some of David Hilbert's work, and began working on abstract algebra.

\item Noether was eventually invited to teach at the prestigious University of G\"ottingen by David Hilbert.
\end{itemize}
\begin{columns}
\begin{column}{0.5\textwidth}
\begin{itemize}
\item Despite the university administration's wishes, she taught there from 1915,
where she was allowed to enter \textit{habilitation} (tenure) in 1919.

\item At G\"ottingen, she produced most of her work on abstract algebra, as
well as proving Noether's Theorem, an important symmetry result in theoretical
physics.
\end{itemize}
\end{column}
\begin{column}{0.5\textwidth}
\includegraphics[width=\textwidth]{gottingen}
\end{column}
\end{columns}
\end{frame}

\subsection{Fate}
\begin{frame}{Fate}
\begin{itemize}
\item Emmy Noether never married or had children.

\item After the Nazi Party's ``Law for the Restoration of the
Professional Civil Service'', she was exiled out of the university.

\item She left for America in 1933 and died two years later in 1935 in surgery
\cite{biography}.
\end{itemize}

\begin{block}{Excerpt from Einstein's Obituary of Emmy Noether}
``Fr\"aulein Noether was the most significant creative mathematical genius thus far produced since the higher education of women began.'' \cite{einstein}
\end{block}
\end{frame}


\section{Mathematical Contributions}
\begin{frame}{Mathematical Contributions}
\begin{block}{Noether's Theorem}
For every physical conservation, there is a differentiable symmetry.

For example, the Lagrangian formulation of classical mechanics provides the corresponding symmetry for the conversation of energy \cite{biography}.
\end{block}

Over the course of her life, she worked on many developing mathematical fields \cite{biography}.
\begin{enumerate}
\item Algebraic invariance
\item Abstract algebra -- primarily theory of rings.
\item Non-commutative algebra and arithmetic.
\end{enumerate}

\begin{alertblock}{Potentially her greatest contribution}
Noether was also known for her consistently invaluable advice. 

Often, she would inspire students with well-placed ideas and improvements,
rarely taking credit for her own work \cite{wikipedia}.
\end{alertblock}
\end{frame}

\section{Interesting Facts} 
\begin{frame}{Interesting Facts}
Striving against the academic norms, Noether pursued a fervent passion in
mathematics, often with a complete disregard for society's restrictions:

\begin{itemize}
    \item Sending postcards continuing trains of thought on mathematics to associate Ernst Fischer.
    \item Becoming so passionate about mathematics as to forget:
    \begin{enumerate}
        \item table-manners
        \item food
        \item frazzled hair
        \item and others.
    \end{enumerate}
    \item Attending classes at a college you cannot attend.
    \item Teaching at a university for no salary.
    \item Allowing on many occasions other people to publish her ideas.
    \item Leaving half of her small salary to her nephew.
\end{itemize}

\begin{alertblock}{Personal Favorite}
Fast-paced lectures often confusing to outsiders. So confusing, one student said of another observer, ``The enemy has been defeated; he has cleared out.'' \cite{biography}.
\end{alertblock}
\end{frame}


\section{Conclusion}
\begin{frame}{Conclusion}
Emmy Noether struggled immensely to leave an impact on the world of mathematics.

\vspace{1em}
Her own work, either directly or indirectly, has reached many diverse fields.

\vspace{1em}
It is likely her unexpected death robbed the world of many further great achievements.

\vspace{1em}
She will be remembered not only for her mathematics, but for
\begin{itemize}
\item persistence
\item courage
\item constant advice
\item and finally as an inspiration to the rest of us.
\end{itemize}
\end{frame}

\begin{frame}{Questions}
\begin{center}
\Huge Questions?
\end{center}
\end{frame}

% You can reveal the parts of a slide one at a time
% with the \pause command:
% \begin{frame}{What makes a good presentation?}
%   \begin{itemize}
%   \item Flow
%   \item Engagement - audience thinks
%   \item Level of difficulty / accessibility. (Background)
%   \item Voice (fluctuation, projection)
%   \item Eye contact
%   \item Figures (labeled, explained)
%   \item Movement, hand gestures
%   \item Amount of math / equations
%   \item Amount of text on slides (they aren't for you to read!) It's a queue.
%   \end{itemize}
% \end{frame}

% \begin{frame}{Actual Slides}
%   \begin{itemize}
%   \item Time : \(15\) min. \(\approx 15\) slides. One minute to each slide.
%   \begin{itemize}
%       \item extra at end to anticipate questions
%       \item 2 minutes for transitions and questions.
%   \end{itemize}
%   \item Use story a narrative. Strife and conflict \(\to\) why should I care?
%   \item Dress? -- \(>\) credibility.
%   \end{itemize}
% \end{frame}

%\section{Second Main Section}

%\subsection{Another Subsection}

% \begin{frame}{Blocks}
% \begin{block}{Block Title}
% You can also highlight sections of your presentation in a block, with it's own title
% \end{block}
% \begin{theorem}
% There are separate environments for theorems, examples, definitions and proofs.
% \end{theorem}
% \begin{example}
% Here is an example of an example block.
% \end{example}
% \end{frame}

% Placing a * after \section means it will not show in the
% outline or table of contents.

\begin{frame}[allowframebreaks]
    \frametitle{References}
    \bibliographystyle{siam}
    \bibliography{new}
\end{frame}


% All of the following is optional and typically not needed. 
%\appendix
%\section<presentation>*{\appendixname}
%\subsection<presentation>*{For Further Reading}

% \begin{frame}[allowframebreaks]
%   \frametitle<presentation>{For Further Reading}
    
%   \begin{thebibliography}{10}
    
%   \beamertemplatebookbibitems
%   % Start with overview books.

%   \bibitem{Author1990}
%     A.~Author.
%     \newblock {\em Handbook of Everything}.
%     \newblock Some Press, 1990.
 
    
%   \beamertemplatearticlebibitems
%   % Followed by interesting articles. Keep the list short. 

%   \bibitem{Someone2000}
%     S.~Someone.
%     \newblock On this and that.
%     \newblock {\em Journal of This and That}, 2(1):50--100,
%     2000.
%   \end{thebibliography}
% \end{frame}

\end{document}